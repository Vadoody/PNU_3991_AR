	
\documentclass[a4]{article}
\usepackage{multicol}
\usepackage{graphicx}
\linespread{1.35}
\usepackage{amsmath}
\usepackage{color}
\usepackage{xcolor}
\usepackage{tikz}
\usepackage{xcolor,colortbl}
\usetikzlibrary{arrows,automata} 



\begin{document}
	
	
	
	\newpage
	
	\vspace*{1cm}
	\begin{center}
		
		{\Huge{\textbf{\textit{$**$ In The Name OF God  $**$}}}}\vspace*{3cm}
		
		
		{\huge{\textbf{\textit{Latex : 621-624}}}}\vspace*{1cm}
		
		
		{\huge{\textbf{\textit{F.Vadoody}}}}\vspace*{1cm}
		
		
		{\huge{\textbf{\textit{Student ID : 949737463}}}}	\vspace*{1cm}
		
		{\huge{\textbf{\textit{Computer Engineering Student}}}}	\vspace*{0.9cm}
		
		{\huge{\textbf{\textit{Majoring In Information Technology}}}}	\vspace*{1cm}
		
		{\huge{\textbf{\textit{Payame Noor University of Tehran, Pardis New City Branch}}}}	\vspace*{1cm}
		
		
	\end{center}
	\vspace*{0.2cm}
	
	\newpage
	\begin{flushright}
		\textcolor{blue}{\hspace*{0.5cm} \texttt{I Advance Topics Related to Automata} \hspace*{0.1cm}\textbf{$|$} \textbf{621}}
	\end{flushright}
	
	\vspace*{0.5cm}
	
	
	Now take a sequence 1001. Let us construct the sequence to $\mathrm{t}_{3}$ using wrap around and null stuffing technique.
	\vspace*{0.2cm}\\
	\large{\textbf{\textit{Wrap Around:}}}
	
	\begin{center}
		\begin{tabular}{lllll}
			\hline
			$\mathrm{t}_{0}$\hspace*{0.7cm}& 1\hspace*{0.7cm} & 0\hspace*{0.7cm} & 0\hspace*{0.7cm} & 1 \\
			$\mathrm{t}_{1}$ & 1 & 1 & 1 & 0 \\
			$\mathrm{t}_{2}$ & 0 & 1 & 1 & 0 \\
			$\mathrm{t}_{3}$ & 1 & 0 & 1 & 1 \\
			\hline
		\end{tabular}
	\end{center}
	
	\vspace*{0.2cm}
	The cells representation are given below 
	\vspace*{0.2cm}
	
	\begin{center}
		\begin{tabular}{|l|l|l|l|}
			\hline
			\cellcolor{black!40}\hspace*{0.9cm}&\hspace*{0.9cm} &\hspace*{0.9cm}&\cellcolor{black!40}\hspace*{0.9cm}\\
			\hline\cellcolor{black!40}&\cellcolor{black!40}& \cellcolor{black!40}&\\
			\hline&\cellcolor{black!40}&\cellcolor{black!40}& \\
			\hline\cellcolor{black!40}& &\cellcolor{black!40}&\cellcolor{black!40} \\
			\hline
		\end{tabular}
	\end{center}
	\vspace*{0.2cm}\\
	\large{\textbf{\textit{Null Stuffed:}}}
	
	In null stuffed two '0' are added at both sides of the original string.
	\vspace*{0.2cm}
	
	\begin{center}
		\begin{tabular}{ccccccc}
			\hline 
			$\mathrm{t}_{0}$\hspace*{0.7cm}& 0\hspace*{0.7cm}& 1\hspace*{0.7cm}& 0\hspace*{0.7cm}& 0\hspace*{0.7cm}& 1\hspace*{0.7cm}& 0\\
			$\mathrm{t}_{1}$\hspace*{0.7cm}& 0\hspace*{0.7cm}& 0\hspace*{0.7cm}& 1\hspace*{0.7cm}& 1\hspace*{0.7cm}& 0\hspace*{0.7cm}& 0\\
			$\mathrm{t}_{2}$\hspace*{0.7cm}& 0\hspace*{0.7cm}& 1\hspace*{0.7cm}& 0\hspace*{0.7cm}& 1\hspace*{0.7cm}& 1\hspace*{0.7cm}& 0\\
			$\mathrm{t}_{3}$\hspace*{0.7cm}& 0\hspace*{0.7cm}& 0\hspace*{0.7cm}& 0\hspace*{0.7cm}& 0\hspace*{0.7cm}& 1\hspace*{0.7cm}& 0\\
			\hline
		\end{tabular}
	\end{center}
	
	\vspace*{0.2cm}
	The cells representation are given below
	
	\begin{center}
		\begin{tabular}{|l|l|l|l|}
			\hline
			\cellcolor{black!40}\hspace*{0.9cm}& \hspace*{0.9cm}&\hspace*{0.9cm}&\cellcolor{black!40}\hspace*{0.9cm}\\
			\hline&\cellcolor{black!40}&\cellcolor{black!40}&\\
			\hline\cellcolor{black!40}& &\cellcolor{black!40}&\cellcolor{black!40}\\
			\hline& & &\cellcolor{black!40}\\
			\hline
		\end{tabular}
	\end{center}
	\vspace*{0.2cm}
	
	
	\fcolorbox{red}{yellow}{\textbf{Example 14.5 :}}\hspace*{0.1cm} \texttt{Find the update of a one-dimensional CA rules for $212 .$}
	\vspace*{0.2cm}\\
	\large{\textbf{\textit{{Solution:}}}
	\vspace*{0.2cm}
	
    The binary equivalent of 212 is $11010100 .$
    The rules are 
    
    \vspace*{0.3cm} 
    \begin{center}
    	\begin{tabular}{lcccccccc}
    		\hline Decimal & 7 & 6 & 5 & 4 & 3 & 2 & 1 & 0 \\
    		Binary & 111 & 110 & 101 & 100 & 011 & 010 & 001 & 000 \\
    		212 & 1 & 1 & 0 & 1 & 0 & 1 & 0 & 0 \\
    	\hline
  	  \end{tabular}
    \end{center}

	\newpage
	\begin{flushleft}
		\textcolor{blue}{\textbf{622}\hspace*{0.1cm} \textbf{$|$} \hspace*{0.1cm} \texttt{Introduction to Automata Theory, Formal Languages and Computation}}
	\end{flushleft}
	\vspace*{0.5cm}
	\begin{flushleft}
		{\Large{\underline{\textbf{14.3.2 Applications of Cellular Automata}}}}
	\end{flushleft}

	\vspace*{0.5cm}
	Cellular automata is not only used in Computer Science field, it  also has different applications in other fields.
	\vspace*{0.2cm}\\
	\large{\textbf{\textit{In the Field of Computer Science}}}
	\vspace*{0.2cm}
		
	\begin{itemize}
		\item Cryptography\vspace*{0.1cm}
		\item Detecting fault tolerance in digital circuit\vspace*{0.1cm}
		\item Simulation of complex system\vspace*{0.1cm}
	\end{itemize}
	\vspace*{0.2cm}\\
	\large{\textbf{\textit{Beyond the Field of Computer Science}}}
	\vspace*{0.2cm}
	
	\begin{itemize}
		\item Simulation of gas behavior\vspace*{0.1cm}
		\item Simulation of forest fire propagation\vspace*{0.1cm}
		\item Simulation of bone erosion.\vspace*{0.1cm}
	\end{itemize}
	Cellular automata are used to identify fault in some digital circuits. 
	Let consider an OR gate\\ 
	Its truth table is\\
	
	\vspace*{0.5cm}

	\begin{center}
		\begin{tabular}{ccc}
			\hline
			\centering
			$\mathrm{X}$ & $\mathrm{Y}$ & $\mathrm{O} / \mathrm{P}$ \\
			\hline 0 & 0 & 0 \\
			0 & 1 & 1 \\
			1 & 0 & 1 \\
			1 & 1 & 1 \\
			\hline
		\end{tabular}
	\end{center}

	\vspace*{0.5cm}
	
	It may happen that the input $Y$ is faulty. It takes ' 1 ' for any input applied to it. Thus the output that we get will always be ' 1 '. This is called 'Struck at 1 '. Similarly the problem 'Struck at 0 ' can occur for a digital circuit.
	\vspace*{0.5cm}\\
	\large{\textbf{\textit{{Rule 192 :}}}
    Binary of 192 is 11000000 

	The rule is
	\vspace*{0.2cm}
	\begin{center}
		\begin{tabular}{ccccccccc}
			\hline & 7 & 6 & 5 & 4 & 3 & 2 & 1 & 0 \\
			& 111 & 110 & 101 & 100 & 011 & 010 & 001 & 000 \\
			192 & 1 & 1 & 0 & 0 & 0 & 0 & 0 & 0 \\
			\hline
		\end{tabular}
	\end{center}
	\vspace*{0.5cm}\\
	Let the initial sequence is 1111 .\\
	By null stuffing at both sides the sequence become 011110 .\\
	So
	\vspace*{0.2cm}
	\begin{center}
		\begin{tabular}{ccccccc}
			\hline
			$\mathrm{t}_{0}$\hspace*{0.7cm}& 0\hspace*{0.7cm}& 1\hspace*{0.7cm}& 1\hspace*{0.7cm}& 1\hspace*{0.7cm}& 1\hspace*{0.7cm}& 0\\
			\hline
			\hline
			$\mathrm{t}_{0}$\hspace*{0.7cm}& 0\hspace*{0.7cm}& 0\hspace*{0.7cm}& 1\hspace*{0.7cm}& 1\hspace*{0.7cm}& 1\hspace*{0.7cm}& 0\\
			\hline
		\end{tabular} 
	\end{center}
	
	\newpage	
	\begin{flushright}
		\textcolor{blue}{\hspace*{0.5cm} \texttt{I Advance Topics Related to Automata} \hspace*{0.1cm}\textbf{$|$} \textbf{623}}
	\end{flushright}

	\vspace*{0.5cm}	
	Let it $t_{0}$ the bits are labeled as $S_{0}, S_{1}, S_{2} \ldots \ldots . S_{5}$ .\\
	State $\mathbf{S}_{1}$ in $\mathrm{t}_{1}$ depends on $\mathbf{S}_{0}, \mathbf{S}_{1}$ and $\mathbf{S}_{2}$ of $\mathbf{t}_{0}$.\\
	The bit pattern is 011 , so the value of $\mathbf{S}_{1}$ in $\mathbf{t}_{1}$ is 0 (according to rule 192 ). By the same way all the other bits are placed. By the same way the patterns for $\mathrm{t}_{2}$ and $\mathrm{t}_{3}$ are generated.
	 
	 
	 \vspace*{0.5cm}
	 \begin{center}
	 	\begin{tabular}{ccccccc}
	 		\hline
	 		$\mathrm{t}_{2}$\hspace*{0.7cm}& 0\hspace*{0.7cm}& 0\hspace*{0.7cm}& 0\hspace*{0.7cm}& 1\hspace*{0.7cm}& 1\hspace*{0.7cm}& 0\\
	 		\hline
	 		\hline
	 		$\mathrm{t}_{3}$\hspace*{0.7cm}& 0\hspace*{0.7cm}& 0\hspace*{0.7cm}& 0\hspace*{0.7cm}& 0\hspace*{0.7cm}& 1\hspace*{0.7cm}& 0\\
	 		\hline
	 	\end{tabular} 
	 \end{center}
 
 
	\vspace*{0.6cm}	

	Let a digital circuit has four inputs.\\
	To identify the ' Struck at 0 ' problem all bits are taken as ' $1$ ' and patterns are generated for next (n-1) steps, where n is the number of input line. If the LSB is other than ' $1$ ' in $t_{n-1}$ step then we can say that the circuit is faulty.
	
	\vspace*{0.9cm}
	
	\newpage
	\begin{flushleft}
		\textcolor{blue}{\textbf{624}\hspace*{0.1cm} \textbf{$|$} \hspace*{0.1cm} \texttt{Introduction to Automata Theory, Formal Languages and Computation}}
	\end{flushleft}
	\vspace*{8cm}
	\begin{center}
	 {\LARGE{\textbf{\textit{This page is intentionally left blank.}}}}
	\end{center}
	\vspace*{8cm}
	
\end{document}
